\documentclass[12pt]{article}
\let\stdsection\section \renewcommand\section{\newpage\stdsection}
\begin{document}

\tableofcontents





\section{Core Functions and Basic Configuration}



\subsection{Core Functions}
\begin{itemize}
\item \texttt{glEnable}, \texttt{glDisable}, \texttt{glIsEnabled} \\
	Controls server-side features.
\item \texttt{glEnableClientState}, \texttt{glDisableClientState} \\
	Controls client-side features.
\item \texttt{glGetBoolean}, \texttt{glGetInteger}, \texttt{glGetFloat}, \texttt{glGetDouble}, \texttt{glGetString} \\
	Reads server-side state variables.
\item \texttt{glFlush} \\
	Flushes commands to the server.
\item \texttt{glFinish} \\
	Flushes commands to the server and waits for them to finish.
\item \texttt{glGetError} \\
	Returns the last error.
\item \texttt{glPushAttrib}, \texttt{glPopAttrib} \\
	Push/pop server-side state.
\item \texttt{glPushClientAttrib}, \texttt{glPopClientAttrib} \\
	Push/pop client-side state.
\end{itemize}



\subsection{Viewport Configuration}
\begin{itemize}
\item \texttt{glViewport} \\
	Defines the transformation from normalized device coordinates to window coordinates.
\item \texttt{glViewportIndexed} \\
	Like \texttt{glViewport}, but takes the index of the viewport to configure.
\item \texttt{glViewportArray} \\
	Like \texttt{glViewport}, but configures multiple viewports at once.
\item \texttt{glScissor} \\
	Defines a \textit{scissor box} in the viewport (using window coordinates) and restricts drawing to that box.
\item \texttt{glScissorIndexed} \\
	Like \texttt{glScissor}, but takes the index of the viewport to configure.
\item \texttt{glScissorArray} \\
	Like \texttt{glScissor}, but configures multiple viewports at once.
\item \texttt{glDepthRange} \\
	Defines the mapping of depth values to internal depth values for the depth buffer.	
\item \texttt{glDepthRangeIndexed} \\
	Like \texttt{glDepthRange}, but takes the index of the viewport to configure.
\item \texttt{glDepthRangeArray} \\
	Like \texttt{glDepthRange}, but configures multiple viewports at once.
\end{itemize}



\subsection{Misc. Configuration}
\begin{itemize}
\item \texttt{glCullFace} \\
	Switches between front-face culling and back-face culling.
\item \texttt{glFrontFace} \\
	Specifies which polygon side is the "front" face.
\item \texttt{glHint} \\
	Sets rendering quality hints.
\end{itemize}





\section{Render State Configuration}



\subsection{Transform Management}

(\texttt{glVertex}) $\rightarrow$ Object Coordinates \\
$\rightarrow$ (ModelView Matrix) $\rightarrow$ Eye Coordinates \\
$\rightarrow$ (Projection Matrix) $\rightarrow$ Clip Coordinates \\
$\rightarrow$ (Divide by w) $\rightarrow$ Normalized Device Coordinates (+-1 cube) \\
$\rightarrow$ (Viewport Transform) $\rightarrow$ Window Coordinates (0,0..w,h)

\subsubsection{General}
\begin{itemize}
\item \texttt{glMatrixMode} \\
	Selects whether the modelview matrix, projection matrix or texture matrix is affected by future calls.
\item \texttt{glPushMatrix} \\
	Pushes the current matrix onto the matrix stack for the current mode (each mode has a separate stack).
\item \texttt{glPopMatrix} \\
	Pops the top of the matrix stack for the current mode.
\end{itemize}

\subsubsection{Loading Matrices}
\begin{itemize}
\item \texttt{glLoadIdentity} \\
	Loads the identity matrix.
\item \texttt{glLoadMatrix} \\
	Loads the specified matrix.
\item \texttt{glLoadTransposeMatrix} \\
	Loads the transpose of the specified matrix.
\end{itemize}

\subsubsection{Multiplying Matrices}
\begin{itemize}
\item \texttt{glTranslate\{f,d\}} \\
	Multiplies the current matrix with a translation matrix.
\item \texttt{glRotate\{f,d\}} \\
	Multiplies the current matrix with a rotation matrix.
\item \texttt{glScale\{f,d\}} \\
	Multiplies the current matrix with a scaling matrix.
\item \texttt{glFrustum} \\
	Multiplies the current matrix with a perspective projection matrix.
\item \texttt{glOrtho} \\
	Multiplies the current matrix with a parallel projection matrix.
\item \texttt{glMultMatrix} \\
	Multiplies the current matrix with the specified matrix.
\item \texttt{glMultTransposeMatrix} \\
	Multiplies the current matrix with the transpose of the specified matrix.
\end{itemize}



\subsection{Rasterizing}

\subsubsection{Color Buffer and Blending}
\begin{itemize}
\item \texttt{glAlphaFunc} \\
	Discards pixels based on comparing the alpha value with a fixed reference value (performance!)
\item \texttt{glBlendFunc}, \texttt{glBlendEquation},  \\
	Defines how pixels are blended into the color buffer.
\item \texttt{glBlendFuncSeparate}, \texttt{glBlendEquationSeparate} \\
	"set the RGB blend equation and the alpha blend equation separately"
\item \texttt{glBlendColor} \\
	Sets the auxiliary color for \texttt{glBlendFunc}.
\item \texttt{glColorMask}, \texttt{glColorMaski} \\
	Blocks off writing to specific color channels.
\end{itemize}

\subsubsection{Depth Buffer}
\begin{itemize}
\item \texttt{glDepthFunc} \\
	Discards pixels based on comparing the depth value with the depth value from the frame buffer (Z-/W-buffering)
\item \texttt{glDepthMask} \\
	Enables / disables writing to the depth buffer.
\end{itemize}

\subsubsection{Stencil Buffer}
\begin{itemize}
\item \texttt{glStencilFunc} \\
	Discards pixels based on comparing a fixed stencil value with the value from the stencil buffer, masking both values with a defined bit mask.
\item \texttt{glStencilMask} \\
	Defines the write-enable bit mask when writing to the stencil buffer.
\item \texttt{glStencilOp} \\
	Defines the effects of (stencil-rejected, stencil-accepted-depth-rejected and both-accepted) pixels on the stencil buffer.
\item \texttt{glStencilFuncSeparate}, \texttt{glStencilMaskSeparate}, \texttt{glStencilOpSeparate} \\
	Configures stencil buffer behavior separately for front-facing and back-facing polygons.
\end{itemize}

\subsubsection{Misc. Rasterizing Configuration}
\begin{itemize}
\item \texttt{glSampleCoverage} \\
	Specifies a bit mask when multisampling each pixel.
\end{itemize}



\subsection{Effects}

\subsubsection{Lighting}
\begin{itemize}
\item \texttt{glLight} \\
	"set light source parameters"
\item \texttt{glGetLight} \\
	"return light source parameter values"
\item \texttt{glLightModel} \\
	"set the lighting model parameters"
\item \texttt{glMaterial} \\
	"specify material parameters for the lighting model"
\item \texttt{glGetMaterial} \\
	"return material parameters values"
\item \texttt{glColorMaterial} \\
	"cause a material color to track the current color"
\item \texttt{glShadeModel} \\
	Switches between flat shading and smooth shading.
\item \texttt{glMinSampleShading} \\
	"specifies minimum rate at which sample shading takes place"
\end{itemize}

\subsubsection{Fog}
\begin{itemize}
\item \texttt{glFog}
\item \texttt{glFogCoord}
\item \texttt{glFogCoordPointer}
\end{itemize}

\subsubsection{Custom Clipping Planes}
\begin{itemize}
\item \texttt{glClipPlane} \\
	Controls custom clipping planes.
\item \texttt{glGetClipPlane} \\
	Controls custom clipping planes.
\end{itemize}

\subsubsection{Decals}
\begin{itemize}
\item \texttt{glPolygonOffset} \\
	"set the scale and units used to calculate depth values" (used for outlining and decals)
\end{itemize}

\subsubsection{Points and Lines}
\begin{itemize}
\item \texttt{glPolygonMode} \\
	Switches between polygon drawing, outline drawing, and vertex drawing.
\item \texttt{glPolygonStipple}
	...
\item \texttt{glGetPolygonStipple}
	...
\item \texttt{glLineWidth} \\
	Specifies the line width for line / outline drawing.
\item \texttt{glLineStipple}
	...
\item \texttt{glPointSize} \\
	Specifies the point size for point / vertex drawing.
\item \texttt{glPointParameter} \\
	Specifies parameters for point / vertex drawing.
\end{itemize}





\section{Primitives, Arrays and Buffers}



\subsection{Direct Mode Rendering}
\begin{itemize}
\item \texttt{glBegin(type)}, \texttt{glEnd} \\
	Start / stop drawing primitives of type \texttt{type}.
\item \texttt{glVertex\{2,3,4\}\{i,f,d\}} \\
	Send a vertex for the current primitive. Default values are (x, y, 0, 1)
\item \texttt{glNormal3\{b,i,f,d\}} \\
	Set the normal for the next vertex.
\item \texttt{glColor\{3,4\}\{ub,b,f,d\}}, \texttt{glColorP\{3,4\}\{u,ui\}} \\
	Set the color for the next vertex.
\item \texttt{glSecondaryColor*} \\
	Sets a secondary per-vertex color that gets added to each pixel.
\item \texttt{glTexCoord\{1,2,3,4\}\{f,d\}} \\
	Set the texture coordinates for the next vertex.
\item \texttt{glMultiTexCoord} \\
	Like \texttt{glTexCoord}, but takes the target texture unit as an additional argument.
\item \texttt{glEdgeFlag} \\
	Hides specific edges when outlining polygons because they're "inner" tesselation edges.
\item \texttt{glRect} \\
	Draws a rectangle. Equivalent to sending four vertices, surrounded by \texttt{glBegin} / \texttt{glEnd}.
\end{itemize}



\subsection{Buffers}

\subsubsection{Life Cycle}
\begin{itemize}
\item \texttt{glGenBuffers} \\
	Generates one or more buffer names.
\item \texttt{glDeleteBuffers} \\
	Deletes one or more buffer names and associated buffers.
\item \texttt{glIsBuffer} \\
	Check if a buffer name is associated with a buffer.
\end{itemize}

\subsubsection{Binding Buffers}
\begin{itemize}
\item \texttt{glBindBuffer} \\
	Binds a buffer to a target for future calls and for drawing, creating the buffer if necessary.
\item \texttt{glBindBufferBase} \\
	bind a buffer object to an indexed buffer target (target with multiple bind points)
\item \texttt{glBindBufferRange} \\
	bind a range within a buffer object to an indexed buffer target (target with multiple bind points)
\end{itemize}
	
\subsubsection{Handling Buffer Data}
\begin{itemize}
\item \texttt{glClearBufferData}, \texttt{glClearBufferSubData} \\
	Fills (part of) a buffer with a fixed value.
\item \texttt{glBufferData} \\
	Defines the usage mode for a buffer and uploads buffer data.
\item \texttt{glBufferSubData} \\
	Updates part of a buffer.
\item \texttt{glMapBuffer}, \texttt{glMapBufferRange}, \texttt{glUnmapBuffer}, \texttt{glGetBufferPointer}, \texttt{glFlushMappedBufferRange} \\
	Maps / unmaps (part of) a buffer to the client's address space; obtains a pointer to the mapped data; flushes changes to a mapped buffer.
\item \texttt{glCopyBufferSubData} \\
	"copy part of the data store of a buffer object to the data store of another buffer object"
\item \texttt{glInvalidateBufferData}, \texttt{glInvalidateBufferSubData} \\
	Invalidates (part of) a buffer.
\item \texttt{glGetBufferSubData} \\
	Reads part of a buffer.
\item \texttt{glGetBufferParameter} \\
	Reads a buffer parameter.
\end{itemize}



\subsection{Using Arrays and Buffers}
(Client-side) arrays and (server-side) buffers are configured using the same functions. For each pointer, if a corresponding
buffer is bound, then it is used (and the "pointer" is an offset for that buffer), otherwise a client-side array is used. Both arrays
and buffers are affected by \texttt{glEnableClientState} and \texttt{glDisableClientState} regarding arrays.

\subsubsection{Configuration}
\begin{itemize}
\item \texttt{glVertexPointer} \\
	Selects a vertex array or sets the offset for a vertex buffer.
\item \texttt{glNormalPointer} \\
	Selects a normal array or sets the offset for a normal buffer.
\item \texttt{glColorPointer} \\
	Selects a per-vertex color array or sets the offset for a per-vertex color buffer.
\item \texttt{glSecondaryColorPointer} \\
	Selects a per-vertex secondary color array or sets the offset for a per-vertex secondary color buffer.
\item \texttt{glTexCoordPointer} \\
	Selects a texture coordinate array or sets the offset for a texture coordinate buffer.
\item \texttt{glEdgeFlagPointer} \\
	Selects a edge flag array or sets the offset for a edge flag buffer.
\item \texttt{glInterleavedArrays} \\
	Deprecated function to set multiple client-side pointers at once.
\item \texttt{glGetPointer} \\
	Returns a client-side array pointer or server-side buffer offset.
\end{itemize}

\subsubsection{Drawing Primitives From Arrays and Buffers}
\begin{itemize}
\item \texttt{glDrawArrays} \\
	Draws primitives from the previously defined arrays.
\item \texttt{glArrayElement} \\
	Draws the vertex from index i of the current array / buffer (requires \texttt{glBegin} / \texttt{glEnd}).
\item \texttt{glDrawElements} \\
	Draws primitives from the previously defined arrays as well as an additional index array that associates primitives with (re-used) vertices.
\item \texttt{glDrawElementsBaseVertex} \\
	Like \texttt{glDrawElements}, with a constant added on-the-fly to each element of the index array.
\item \texttt{glDrawRangeElements} \\
	Like \texttt{glDrawElements} but with additional lower and upper bound hints for the indices from the index array.
\item \texttt{glDrawRangeElementsBaseVertex} \\
	Like \texttt{glDrawRangeElements}, with a constant added on-the-fly to each element of the index array.
\item \texttt{glMultiDrawArrays}, \texttt{glMultiDrawElements} \\
	Performs multiple \texttt{glDrawArrays} / \texttt{glDrawElements} invocations at once, taking the parameters for each
	invocation from an array.
\item \texttt{glPrimitiveRestartIndex}
	Sets the "primitive restart index", that is, an array index that doesn't issue a vertex but instead starts a new draw operation.
\end{itemize}

\subsubsection{Instancing Support for Shaders}
These functions repeat the above functions n times, incrementing a counter each time. They are useless without a
geometry shader that interprets the counter, since they otherwise just repeatedly draw the same object. A shader could,
for example, apply a different transformation each time to draw multiple instances at different locations.
\begin{itemize}
\item \texttt{glDrawArraysInstanced} \\
	Repeats \texttt{glDrawArrays} n times.
\item \texttt{glDrawElementsInstanced} \\
	Repeats \texttt{glDrawElements} n times.
\item \texttt{glDrawArraysInstancedBaseInstance} \\
	Like \texttt{glDrawArraysInstanced}, but respects "instanced" generic vertex attributes.
\item \texttt{glDrawElementsInstancedBaseInstance} \\
	Like \texttt{glDrawElementsInstanced}, but respects "instanced" generic vertex attributes.
\item \texttt{glDrawElementsInstancedBaseVertex} \\
	Like \texttt{glDrawElementsInstanced}, with a constant added on-the-fly to each element of the index array.
\item \texttt{glDrawElementsInstancedBaseVertexBaseInstance} \\
	Like \texttt{glDrawElementsInstanced}, with a constant added on-the-fly to each element of the index array AND
	respects "instanced" generic vertex attributes.
\item \texttt{glDrawArraysIndirect} \\
	Variant of \texttt{glDrawArraysInstancedBaseInstance} that takes a pointer to the remaining function arguments.
\item \texttt{glDrawElementsIndirect} \\
	Variant of \texttt{glDrawElementsInstancedBaseVertexBaseInstance} that takes a pointer to the remaining function
	arguments.
\item \texttt{glMultiDrawArraysIndirect}, \texttt{glMultiDrawElementsIndirect} \\
	Performs multiple \texttt{glDrawArraysInstancedBaseInstance} / \texttt{glDrawElementsInstancedBaseVertexBaseInstance}
	invocations at once, taking the parameters for each invocation from an array.
\end{itemize}





\section{Textures, Images and Bitmaps}



\subsection{Dealing With Image Data}

\subsubsection{Pixel Handling Configuration}
Affects texture uploading, texture downloading, drawing pixels and bitmaps, etc. 
\begin{itemize}
\item \texttt{glPixelStore\{i,f\}}
	Defines how pixels are stored in application memory.
\item \texttt{glPixelTransfer\{i,f\}}
	Defines a per-channel scale and offset when reading, writing or copying pixels.
\item \texttt{glPixelMap\{usv,uiv,fv\}}
	Defines a per-channel mapping, e.g. for palette lookup or gamma correction.
\item \texttt{glGetPixelMap}
	Reads the pixel map.
\end{itemize}



\subsection{Handling Textures}

\subsubsection{Life Cycle}
\begin{itemize}
\item \texttt{glGenTextures} \\
	Generates one or more texture names.
\item \texttt{glDeleteTextures} \\
	Deletes texture names and associated textures.
\item \texttt{glIsTexture} \\
	Checks whether a texture name is associated with a texture.
\end{itemize}

\subsubsection{Binding Textures}
\begin{itemize}
\item \texttt{glBindTexture} \\
	Binds a texture for future calls and for drawing. The target for the first binding of a texture determines its dimension.
	Separate bindings exist per target.
\end{itemize}

\subsubsection{Configuring Textures}
\begin{itemize}
\item \texttt{glTexStorage*}, \texttt{glTexStorage*Multisample} \\
	Specifies the storage for all levels of a texture at once.
\item \texttt{glGenerateMipmap} \\
	"generate mipmaps for a specified texture target"
\end{itemize}

\subsubsection{Handling Texture Data}
\begin{itemize}
\item \texttt{glTexImage\{1D,2D,3D\}}, \texttt{glTexImage\{2D,3D\}Multisample} \\
	Upload texture image data.
\item \texttt{glTexSubImage\{1D,2D,3D\}} \\
	Update part of a texture image.
\item \texttt{glCompressedTexImage\{1D,2D,3D\}} \\
	Upload texture image data from a compressed source.
\item \texttt{glCompressedTexSubImage\{1D,2D,3D\}} \\
	Update part of a texture image from a compressed source.
\item \texttt{glInvalidateTexImage}, \texttt{glInvalidateTexSubImage} \\
	Invalidate (part of) a texture.
\item \texttt{glGetTexImage} \\
	Download texture image data.
\item \texttt{glGetCompressedTexImage} \\
	"return a compressed texture image"
\item \texttt{glTextureView} \\
	"initialize a texture as a data alias of another texture's data store"
\item \texttt{glCopyImageSubData} \\
	"perform a raw data copy between two images"
\end{itemize}

\subsubsection{Texture-Related Render State}
\begin{itemize}
\item \texttt{glActiveTexture}, \texttt{glClientActiveTexture} \\
	Selects the active texture unit. The server-side active unit affects future server-side operations and drawing; the client-side
	active unit affects future client-side operations (e.g. \texttt{glTexCoordPointer}).
\item \texttt{glTexParameter(target, parameterName, parameterValue} \\
	Set texture parameters.
\item \texttt{glGetTexParameter	} \\
	Get texture parameters.
\item \texttt{glGetTexLevelParameter(target, level, parameterName, parameterValue} \\
	Get LOD-(mipmap-)dependent texture parameters.
\item \texttt{glTexEnv} \\
	???
\item \texttt{glGetTexEnvf} \\
	???
\item \texttt{glTexGen} \\
	Defines automatic generation of texture coordinates.
\item \texttt{glGetTexGen} \\
	Returns automatic texture generation parameters.
\item \texttt{glAreTexturesResident} \\
	"determine if textures are loaded in texture memory"
\end{itemize}



\subsection{Samplers}

\subsubsection{Life Cycle}
\begin{itemize}
\item \texttt{glGenSamplers} \\
	"generate sampler object names"
\item \texttt{glDeleteSamplers} \\
	"delete named sampler objects"
\item \texttt{glIsSampler} \\
	"determine if a name corresponds to a sampler object"
\end{itemize}

\subsubsection{Binding Samplers}
\begin{itemize}
\item \texttt{glBindSampler} \\
	"bind a named sampler to a texturing target"
\end{itemize}

\subsubsection{Configuring Samplers}
\begin{itemize}
\item \texttt{glSamplerParameter} \\
	Specifies sampler parameters.
\item \texttt{glGetSamplerParameter} \\
	Returns sampler parameters.
\end{itemize}



\subsection{Connecting Textures and Buffers}
\begin{itemize}
\item \texttt{glTexBuffer} \\
	"attach the storage for a buffer object to the active buffer texture"
\item \texttt{glTexBufferRange} \\
	"bind a range of a buffer's data store to a buffer texture"
\end{itemize}



\subsection{Drawing Bitmaps and Pixels}
\begin{itemize}
\item \texttt{glRasterPos\{2,3,4\}\{s,i,f,d\}} \\
	Sets the raster position by transforming 3d coordinates like for \texttt{glVertex}.
\item \texttt{glWindowPos} \\
	Sets the raster position directly using window coordinates.
\item \texttt{glBitmap} \\
	Draws a bitmap, with a "1" bit using the current color and a "0" bit being transparent, to the current raster position.
	This function also updates the raster position.
\item \texttt{glDrawPixels} \\
	Copies pixels from memory to the current raster position of the framebuffer.
\item \texttt{glCopyPixels} \\
	Copies pixels from the specified framebuffer position to the current raster position.
\item \texttt{glPixelZoom}
	Specifies the zoom factor for \texttt{glDrawPixels} and \texttt{glCopyPixels}.
\end{itemize}





\section{OpenGL (non-shader part)}

\subsection{Display Lists}
\begin{itemize}
\item \texttt{glGenLists} \\
	Generates display lists.
\item \texttt{glDeleteLists} \\
	Deletes display lists.
\item \texttt{glIsList} \\
	"determine if a name corresponds to a display list"
\item \texttt{glNewList} \\
	Start filling a display list.
\item \texttt{glEndList} \\
	Finish filling a display list.
\item \texttt{glCallList} \\
	Executes a display list.
\item \texttt{glCallLists} \\
	Executes multiple display lists.
\item \texttt{glListBase} \\
	Defines an offset for display list indices when calling lists.
\end{itemize}




\subsection{Splines, B-Splines, NURBS, Bezier Curves, etc.}
\begin{itemize}
\item \texttt{glMapGrid\{1,2\}\{f,d\}}
\item \texttt{glMap\{1,2\}\{f,d\}}
\item \texttt{glGetMap\{1,2\}\{f,d\}}
\item \texttt{glEvalPoint\{1,2\}}
\item \texttt{glEvalCoord\{1,2\}\{f,d,fv,dv\}}
\item \texttt{glEvalMesh\{1,2\}}
\end{itemize}


\subsection{Synchronization}
Used to synchronize client and server processes.
\begin{itemize}
\item \texttt{glClientWaitSync}
\item \texttt{glFenceSync}
\item \texttt{glIsSync}
\item \texttt{glWaitSync}
\item \texttt{glGetSync}
\item \texttt{glDeleteSync}
\item \texttt{glObjectPtrLabel}
\item \texttt{glGetObjectPtrLabel}
\item \texttt{glMemoryBarrier}
\end{itemize}




\subsection{Frame Buffer Management}
\begin{itemize}
\item \texttt{glClear} \\
	Clears one or more buffers.
\item \texttt{glClearColor} \\
	Set the value that \texttt{glClear} will use for the color buffer.
\item \texttt{glClearDepth} \\
	Set the value that \texttt{glClear} will use for the depth buffer.
\item \texttt{glClearStencil} \\
	Set the value that \texttt{glClear} will use for the stencil buffer.
\item \texttt{glClearAccum} \\
	Set the value that \texttt{glClear} will use for the accumulation buffer.
\item \texttt{glAccum} \\
	Transfer between color buffer and accumulation buffer.
\item \texttt{glDrawBuffer} \\
	Selects the buffer (double buffering, stereo rendering)
\item \texttt{glInvalidateFramebuffer}, \texttt{glInvalidateSubFramebuffer} \\
	Invalidates (part of) the frame buffer.
\item \texttt{glGenFramebuffers} \\
	...
\item \texttt{glDeleteFramebuffers} \\
	...
\item \texttt{glBindFramebuffer} \\
	...
\item \texttt{glIsFramebuffer} \\
	...
\item \texttt{glCheckFramebufferStatus} \\
	...
\item \texttt{glFramebufferTexture} \\
	"attach a level of a texture object as a logical buffer to the currently bound framebuffer object"
\item \texttt{glFramebufferTextureLayer} \\
	"attach a single layer of a texture to a framebuffer"
\item \texttt{glClearBuffer} \\
	"clear individual buffers of the currently bound draw framebuffer"
\item \texttt{glFramebufferParameter} \\
	"set a named parameter of a framebuffer"
\item \texttt{glGetFramebufferAttachmentParameter} \\
	"retrieve information about attachments of a bound framebuffer object"
\item \texttt{glFramebufferParameter} \\
	"set a named parameter of a framebuffer"
\item \texttt{glGetFramebufferParameter} \\
	"retrieve a named parameter from a framebuffer"
\item \texttt{glGetMultisample} \\
	"retrieve the location of a sample"
\end{itemize}

\subsection{Renderbuffers}
\begin{itemize}
\item \texttt{glGenRenderbuffers} \\
\item \texttt{glDeleteRenderbuffers} \\
\item \texttt{glIsRenderbuffer} \\
\item \texttt{glBindRenderbuffer} \\
\item \texttt{glGetRenderbufferParameter} \\
\item \texttt{glRenderbufferStorage} \\
\item \texttt{glRenderbufferStorageMultisample} \\
\item \texttt{glFramebufferRenderbuffer} \\
	"attach a renderbuffer as a logical buffer to the currently bound framebuffer object"
\item \texttt{glSampleMaski} \\
	"set the value of a sub-word of the sample mask"
\end{itemize}

\subsection{Reading the Frame Buffer}
\begin{itemize}
\item \texttt{glReadBuffer} \\
	Selects the frame buffer to read from.
\item \texttt{glCopyTexImage\{1,2\}D} \\
	Defines a texture by copying a rectangular part of the frame buffer.
\item \texttt{glCopyTexSubImage\{1,2,3\}D} \\
	Updates a texture by copying a rectangular part of the frame buffer.
\item \texttt{glReadPixels} \\
	Copies pixels from the framebuffer into memory.
\item \texttt{glClampColor} \\
	"specify whether data read via glReadPixels should be clamped"
\item \texttt{glBlitFramebuffer} \\
	Copies pixels within the frame buffer or from one frame buffer to another.
\end{itemize}

\subsection{Render Queries}
\begin{itemize}
\item \texttt{glGenQueries} \\
	"generate query object names"
\item \texttt{glDeleteQueries} \\
	"delete named query objects"
\item \texttt{glIsQuery} \\
	"determine if a name corresponds to a query object"
\item \texttt{glGetQueryObject} \\
	"return parameters of a query object"
\item \texttt{glGetQueryIndexed} \\
	"return parameters of an indexed query object target"
\item \texttt{glBeginQuery}, \texttt{glEndQuery} \\
	"delimit the boundaries of a query object"
\item \texttt{glBeginQueryIndexed}, \texttt{glEndQueryIndexed} \\
	"delimit the boundaries of a query object on an indexed target"
\item \texttt{glGetQuery} \\
	"return parameters of a query object target"
\item \texttt{glQueryCounter} \\
	"record the GL time into a query object after all previous commands have reached the GL server but have not yet necessarily executed"
\item \texttt{glBeginConditionalRender}, \texttt{glEndConditionalRender} \\
	Execute GL commands only if a query detects that samples passed the tests.
\end{itemize}

\subsection{Debugging}
\begin{itemize}
\item \texttt{glDebugMessageControl} \\
	"control the reporting of debug messages in a debug context"
\item \texttt{glPushDebugGroup} \\
	"push a named debug group into the command stream"
\item \texttt{glPopDebugGroup} \\
	"pop the active debug group"
\item \texttt{glDebugMessageCallback} \\
	"specify a callback to receive debugging messages from the GL"
\item \texttt{glDebugMessageInsert} \\
	"inject an application-supplied message into the debug message queue"
\item \texttt{glGetDebugMessageLog} \\
	"retrieve messages from the debug message log"
\item \texttt{glObjectLabel} \\
	"label a named object identified within a namespace"
\item \texttt{glGetObjectLabel} \\
	"retrieve the label of a named object identified within a namespace"
\end{itemize}

\subsection{Feedback Buffers (structured rendering log, selection)}
\begin{itemize}
\item \texttt{glInitNames} \\
	...
\item \texttt{glLoadName} \\
	...
\item \texttt{glPushName} \\
	...
\item \texttt{glPopName} \\
	...
\item \texttt{glFeedbackBuffer} \\
	...
\item \texttt{glSelectBuffer} \\
	...
\item \texttt{glRenderMode} \\
	...
\item \texttt{glPassThrough} \\
	...
\end{itemize}

\subsection{Various Rendering Parameters}
\begin{itemize}
\item \texttt{glPrioritizeTextures} \\
	Sets texture priority with respect to eviction from the graphics memory.
\item \texttt{glGetInternalformat} \\
	"retrieve information about implementation-dependent support for internal formats"
\item \texttt{glPatchParameter} \\
	"specifies the parameters for patch primitives" (hardware tesselation control)
\end{itemize}

\section{OpenGL Shaders}

\begin{itemize}
\item \texttt{glCreateShader}
\item \texttt{glShaderSource}
\item \texttt{glCompileShader}
\item \texttt{glCreateProgram}
\item \texttt{glAttachShader}
\item \texttt{glLinkProgram}
\item \texttt{glUseProgram}
\item \texttt{glProgramUniform}, \texttt{glProgramUniformMatrix}
\item \texttt{glGetProgramiv}
\item \texttt{glValidateProgram}
\item \texttt{glValidateProgramPipeline}
\item \texttt{glGetProgramBinary}
\item \texttt{glGetProgramResourceLocation}
\item \texttt{glGetProgramStage}
\item \texttt{glGetProgram}
\item \texttt{glCreateShaderProgram}
\item \texttt{glGetProgramPipeline.}
\item \texttt{glGetProgramPipelineInfoLog}
\item \texttt{glGetProgramInterface.}
\item \texttt{glIsProgram}
\item \texttt{glDeleteProgramPipelines}
\item \texttt{glDeleteProgram}
\item \texttt{glGetProgramResourceIndex}
\item \texttt{glGetProgramResourceLocationIndex}
\item \texttt{glActiveShaderProgram}
\item \texttt{glGetProgramStagei}
\item \texttt{glBindProgramPipeline}
\item \texttt{glGetProgramResource}
\item \texttt{glGetProgramResourceName}
\item \texttt{glGenProgramPipelines}
\item \texttt{glGetProgramInfoLog}
\item \texttt{glProgramParameteri}
\item \texttt{glUseProgramStages}
\item \texttt{glProgramBinary}
\item \texttt{glIsProgramPipeline}
\item \texttt{glGetShaderInfoLog}
\item \texttt{glDeleteShader}
\item \texttt{glDetachShader}
\item \texttt{glShaderBinary}
\item \texttt{glGetAttachedShaders}
\item \texttt{glGetShader}
\item \texttt{glGetShaderSource}
\item \texttt{glGetShaderPrecisionFormat}
\item \texttt{glShaderStorageBlockBinding}
\item \texttt{glIsShader}
\item \texttt{glGetShaderi}
\item \texttt{glReleaseShaderCompiler}
\item \texttt{glBindAttribLocation}
\item \texttt{glGetUniform}
\item \texttt{glGetUniformLocation}
\item \texttt{glGetActiveAttrib}
\item \texttt{glGetActiveAttribSize}
\item \texttt{glGetActiveAttribType}
\item \texttt{glGetActiveUniform}
\item \texttt{glGetAttribLocation}
\item \texttt{glVertexAttrib}
\item \texttt{glGetVertexAttrib}
\item \texttt{glVertexAttribPointer}
\item \texttt{glGetVertexAttribPointer}
\item \texttt{glUniform}
\item \texttt{glUniformMatrix}
\item \texttt{glUniformSubroutinesu}
\item \texttt{glGetUniformBlockIndex}
\item \texttt{glGetActiveSubroutineUniform}
\item \texttt{glGetActiveSubroutineUniformName}
\item \texttt{glGetActiveUniformType}
\item \texttt{glGetActiveUniformSize}
\item \texttt{glGetActiveUniformBlockName}
\item \texttt{glGetSubroutineUniformLocation}
\item \texttt{glGetUniformIndices}
\item \texttt{glGetActiveUniforms}
\item \texttt{glUniformBlockBinding}
\item \texttt{glGetActiveUniformBlock}
\item \texttt{glGetActiveUniformName}
\item \texttt{glGetSubroutineIndex}
\item \texttt{glGetActiveSubroutineName}
\item \texttt{glVertexAttribFormat}
\item \texttt{glVertexAttribDivisor}
\item \texttt{glVertexAttribBinding}
\item \texttt{glEnableVertexAttribArray}
\item \texttt{glDisableVertexAttribArray}
\item \texttt{glDrawBuffers}
\item \texttt{glVertexBindingDivisor}
\item \texttt{glBindFragDataLocation}
\item \texttt{glBindFragDataLocationIndexed}
\item \texttt{glGetActiveAtomicCounterBuffer}
\item \texttt{glBindImageTexture}
\item \texttt{glGetUniformSubroutine}
\item \texttt{glDispatchCompute}
\item \texttt{glDispatchComputeIndirect}
\item \texttt{glProvokingVertex}
\item \texttt{glGetFragDataIndex}
\item \texttt{glGetFragDataLocation}
\item \texttt{glBindVertexBuffer} \\
	"bind a buffer to a vertex buffer bind point" (only used for generic vertex attributes)
\end{itemize}

\subsection{Vertex Array Objects (related to generic vertex attributes)}
\begin{itemize}
\item \texttt{glGenVertexArrays} \\
	"generate vertex array object names"
\item \texttt{glDeleteVertexArrays} \\
	"delete vertex array objects"
\item \texttt{glIsVertexArray} \\
	"determine if a name corresponds to a vertex array object"
\item \texttt{glBindVertexArray} \\
	"bind a vertex array object"
\end{itemize}

\subsection{Transform Feedback}
\begin{itemize}
\item \texttt{glGenTransformFeedbacks}
\item \texttt{glDeleteTransformFeedbacks}
\item \texttt{glBeginTransformFeedback}
\item \texttt{glEndTransformFeedback}
\item \texttt{glBindTransformFeedback}
\item \texttt{glDrawTransformFeedback}
\item \texttt{glDrawTransformFeedbackInstanced}
\item \texttt{glDrawTransformFeedbackStream}
\item \texttt{glDrawTransformFeedbackStreamInstanced}
\item \texttt{glGetTransformFeedbackVarying}
\item \texttt{glIsTransformFeedback}
\item \texttt{glPauseTransformFeedback}
\item \texttt{glResumeTransformFeedback}
\item \texttt{glTransformFeedbackVaryings}
\end{itemize}

\section{Useless OpenGL Features}

\subsection{Functions Related to Indexed Color Modes}
\begin{itemize}
\item \texttt{glIndexPointer}
\item \texttt{glClearIndex}
\item \texttt{glLogicOp}
\end{itemize}

\end{document}
